\usepackage[left=1.5cm,right=1.5cm,top=2cm,bottom=2cm]{geometry}
\usepackage[utf8]{inputenc}		        % Accents, encodage utf8
\usepackage[T1]{fontenc}		        		% Encodage des caractères
\usepackage{lmodern}			        		% Choix de la fonte (Latin Modern de D. Knuth)
\usepackage[francais]{babel}	        		% Les règles de typo. françaises
\usepackage{multicol} 					% Multi-colonnes
\usepackage{calc} 						% Calculs 
\usepackage{enumerate}					% Pour modifier les numérotations
\usepackage{enumitem}
\usepackage{graphicx}					% Pour insérer des images
\usepackage{tabularx}					% Pour faire des tableaux
\usepackage{pgf,tikz}					% Pour les images et figures géométriques
\usetikzlibrary{arrows,calc,fit,patterns,plotmarks,shapes.geometric,shapes.misc,shapes.symbols,shapes.arrows,
shapes.callouts, shapes.multipart, shapes.gates.logic.US,shapes.gates.logic.IEC, er, automata,backgrounds,chains,topaths,trees,petri,mindmap,matrix, calendar, folding,fadings,through,positioning,scopes,decorations.fractals,decorations.shapes,decorations.text,decorations.pathmorphing,decorations.pathreplacing,decorations.footprints,decorations.markings,shadows,babel} % Charge toutes les librairies de Tikz

\usepackage{tkz-tab,tkz-euclide,tkz-fct}	% Géométrie euclidienne avec TikZ

\usetkzobj{all}				
		
\usepackage{amsmath,amsfonts,amssymb,mathrsfs}  % Spécial math
\usepackage[squaren]{SIunits}			% Pour les unités (gère le conflits avec  \square de l'extension amssymb)
\usepackage{pifont}						% Pour les symboles "ding"
\usepackage{bbding}						% Pour les symboles
\usepackage[misc]{ifsym}					% Pour les symboles
\usepackage{cancel}						% Pour pouvoir barrer les nombres
\usepackage{url} 			        		% Pour afficher correctement les url
 \urlstyle{sf}                          	% qui s'afficheront en police sans serif
\usepackage{eurosym}					% Pour utiliser la commande \euro
\usepackage{fancyhdr,lastpage}          	% En-têtes et pieds
 \pagestyle{fancy}                      	% de pages personnalisés
\usepackage{fancybox}					% Pour les encadrés
\usepackage{xlop}						% Pour les calculs posés
\usepackage{standalone}					% Pour avoir un apercu d'un fichier qui sera utilisé avec un input
\usepackage{multido}					% Pour faire des boucles
\usepackage{hyperref}					% Pour gérer les liens hyper-texte
\usepackage{fourier}
\usepackage{colortbl} 					% Pour des tableaux en couleur
\usepackage{setspace}					% Pour \begin{spacing}{2.0} \end{spacing}
\usepackage{multirow}					% Pour des cellules multilignes dans un tableau
\usepackage{import}						% Equivalent de input mais en spécifiant le répertoire de travail
\usepackage[]{qrcode}
\usepackage{pdflscape}
\usepackage[framemethod=tikz]{mdframed} % Pour les cadres
\usepackage{tikzsymbols}
\usepackage{tasks}						% Pour les listes horizontales


				
\setlength{\parindent}{0mm}				% Pas de retrait en début de paragraphe

\renewcommand{\arraystretch}{1.5}		% Interligne dans les tableaux


\newcounter{exo}          				% déclaration du numéro d'exo (pas utilisé ici)
\setcounter{exo}{0}   					% initialisation du numero
\newcommand{\exo}[1]{					% \exo
  	\stepcounter{exo}        			% incrémentation du numéro
  	\subsection*{Exercice n°{\theexo} \textmd{\normalsize #1}}
  	}
  	
\renewcommand{\labelenumi}{\textbf{\theenumi{}.}}			% Numérotation en gras
\renewcommand{\labelenumii}{\textbf{\theenumii{}.}}		% Numérotation de niveau 2 en gras
\renewcommand{\thesection}{\Roman{section}.}				% Numérotation des sections en chiffres romains
\renewcommand{\thesubsection}{\alph{subsection})}			% Numérotation des sous-sections en lettres

\newcommand{\version}[1]{\fancyhead[R]{Version #1}}

\setlength{\fboxsep}{3mm}

\graphicspath{{./images/}}


\newmdenv[startcode={\setcounter{exo}{0}},linecolor=red,roundcorner=10pt, linewidth=3pt,frametitlerule=true,frametitlebackgroundcolor=gray!20,apptotikzsetting={\tikzset{mdfframetitlebackground/.append style={shade,left color=gray!20, right color=gray!80}}},frametitlerulewidth=1pt,frametitle=Corrections,innerbottommargin=10pt,innertopmargin=10pt,everyline=true]{correction}





