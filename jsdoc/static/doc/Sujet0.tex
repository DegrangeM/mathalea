\documentclass[12pt,a4paper]{article}

\usepackage{csvsimple}%

\usepackage[francais,bloc,completemulti,insidebox]{automultiplechoice}
 \usepackage{geometry} 
   \geometry{headsep=0.3cm, left=1.5cm,right=1.5cm,top=2.4cm,bottom=1.5cm}
   
   \AMCcodeHspace=.3em % réduction de la taille des cases pour le code élève
   \AMCcodeVspace=.3em 
  % \AMCcodeBoxSep=.1em
   
   \def\AMCotextReserved{\emph{Ne rien cocher, réservé au prof !}}
   \AMCboxStyle{shape=square,size=2.5ex,down=.4ex,rule=0.5pt,outsidesep=.1em,color=black}
  %%%%%% Définition des barèmes 
  \baremeDefautS{
    e=0.0001,% incohérence (plusieurs réponses données à 0,0001 pour définir des manquements au respect de consignes)
    b=1,% bonne réponse 1
    m=-0.01,% mauvaise réponse 0,01 pour différencier de la 
    v=0} % non réponse qui reste à 0
  
  \baremeDefautM{formula=((NBC-NMC)/NB)*((NBC-NMC)/NB>0)} % nombre de bonnes réponses cochées minorées des mauvaises réponses cochées, ramenées à 1, et ramenée à 0 si résultat négatif.
  
  %%%%%%%%% Paramètres pour réponses à construire 
  \AMCinterIrep=0pt \AMCinterBrep=.5ex \AMCinterIquest=0pt \AMCinterBquest=3ex \AMCpostOquest=7mm \setlength{\AMChorizAnswerSep}{3em plus 4em} \setlength{\AMChorizBoxSep}{1em}

\newcommand{\sujet}{
\exemplaire{1}{%

%%% debut de l'en-tête des copies :  
\begin{center}
\noindent{}\fbox{\vspace*{3mm}
         \Large\bf\nom{}~\prenom{}\normalsize{}% 
          \vspace*{3mm}
      }
\end{center}

\noindent{\bf QCM  \hfill TEST}

\vspace*{.5cm}
\begin{minipage}{.4\linewidth}
  \centering\large\bf Test\\ Examen du 01/01/2008
\end{minipage}

\begin{center}\em
Durée : 10 minutes.

  Aucun document n'est autorisé.
  L'usage de la calculatrice est interdit.

  Les questions faisant apparaître le symbole \multiSymbole{} peuvent
  présenter zéro, une ou plusieurs bonnes réponses. Les autres ont
  une unique bonne réponse.

  Des points négatifs pourront être affectés à de \emph{très
    mauvaises} réponses.
\end{center}
\vspace{1ex}
%%% fin de l'en-tête

\restituegroupe{general}
 


\AMCassociation{\id}

%\AMCaddpagesto{3}
	  }
}

%%%%§§§§§§§§§§§§§§§§§§§§§§§§§§§§§§§§§

\begin{document}
%%%Options
\AMCrandomseed{1237893}

\def\AMCformQuestion#1{{\sc Question #1 :}}

\setdefaultgroupmode{withoutreplacement}
%%% Fin Options

%%% groupes


\element{general}{
  \begin{question}{qcm1}    
    Coche la bonne réponse et uniquement celle-là.
    \begin{reponses}
      \bonne{Bonne réponse}
      \mauvaise{Mauvaise réponse}
      \mauvaise{Mauvaise réponse}
      \mauvaise{Mauvaise réponse}
    \end{reponses}
  \end{question}
}

\element{general}{
  \begin{questionmult}{qcm2}    
    Coche les bonnes réponses. (le symbole \multiSymbole{} signale une question qui peut avoir plusieurs bonnes réponses
    \begin{reponses}
      \bonne{Bonne réponse}
      \mauvaise{Mauvaise réponse}
      \bonne{Bonne réponse}
    \end{reponses}
  \end{questionmult}
}

\element{general}{
  \begin{questionmult}{qcm3}
    Coche la ou les bonnes réponses (le symbole \multiSymbole{} signale une question qui peut avoir plusieurs bonnes réponses mais pas forcément plusieurs)
    \begin{reponseshoriz}[o]
      \mauvaise{Mauvaise réponse}
      \mauvaise{Mauvaise réponse}
      \mauvaise{Mauvaise réponse}
      \bonne{Bonne réponse}
      \mauvaise{Mauvaise réponse}
    \end{reponseshoriz}
  \end{questionmult}
}

\element{general}{
  \begin{questionmult}{qcm4}
    Coche la ou les bonnes réponses (le symbole \multiSymbole{} signale une question qui peut avoir plusieurs bonnes réponses mais qui n'en a pas forcément)
    \begin{reponseshoriz}[o]
      \mauvaise{Mauvaise réponse}
      \mauvaise{Mauvaise réponse}
      \mauvaise{Mauvaise réponse}
      \mauvaise{Mauvaise réponse}
      \mauvaise{Mauvaise réponse}
    \end{reponseshoriz}
  \end{questionmult}
}
\element{general}{
 \begin{questionmultx}{num1} 
 Coder le nombre 47 avec les cases ci-dessous (un chiffre par ligne, les dizaines puis les unités). 
 \AMCnumericChoices{47}{digits=2,decimals=0,sign=false,borderwidth=0pt,backgroundcol=lightgray,scoreapprox=0.5,scoreexact=1,Tpoint={,}}
\end{questionmultx}
 }
 
 \element{general}{
 \begin{questionmultx}{num2} 
 Coder le nombre 8 avec les cases ci-dessous (un chiffre par ligne, 0 dizaine et 8 unités). 
 \AMCnumericChoices{8}{digits=2,decimals=0,sign=false,borderwidth=0pt,backgroundcol=lightgray,scoreapprox=0.5,scoreexact=1,Tpoint={,}}
\end{questionmultx}
 }
 
 \element{general}{
 \begin{questionmultx}{num3} 
 Coder le nombre 2,3 avec les cases ci-dessous (un chiffre par ligne, les unités avant la virgule les dixièmes ensuite). 
 \AMCnumericChoices{2.3}{digits=2,decimals=1,sign=false,borderwidth=0pt,backgroundcol=lightgray,scoreapprox=0.5,scoreexact=1,Tpoint={,}}
\end{questionmultx}
 }
 
 
 \element{general}{
 \begin{questionmultx}{num4} 
 Coder le nombre $-1,2\times 10^{-3}$ avec les cases ci-dessous (un chiffre par ligne, et n'oublie pas de coder les signes moins). 
 \AMCnumericChoices{-0.0012}{digits=2,decimals=1,sign=true,exponent=1,exposign=true,borderwidth=0pt,backgroundcol=lightgray,scoreapprox=0.5,scoreexact=1,Tpoint={,}}
\end{questionmultx}
 }

\element{general}{
\begin{question}{open1}
dessine un triangle ABC dans le cadre ci-dessous
\AMCOpen{lines=4,dots=false}{\mauvaise[{\tiny NR}]{NR}\scoring{0}\mauvaise[{\tiny RR}]{RR}\scoring{0.01}\mauvaise[{\tiny R}]{R}\scoring{0.33}\mauvaise[{\tiny V}]{V}\scoring{0.67}\bonne[{\tiny VV}]{VV}\scoring{1}}
\end{question}
}

%%%% fin des groupes


\csvreader[head to column names]{liste.csv}{}{\sujet}

\end{document}
