%On souhaite réaliser une frise composée de rectangles. 
%
%Pour cela, on a écrit le programme ci-dessous:
%
%\begin{center}
%\begin{tabularx}{\linewidth}{|X|X|}\hline
%\begin{scratch}
%\blockinit{quand \greenflag est cliqué}
%\blockcontrol{cacher}
%\blockpen{mettre la taille du stylo à \ovalnum{1}}
%\blockpen{effacer tout}
%\blockmove{aller à x: \ovalnum0 y: \ovalnum0}
%\blockrepeat{répéter \ovalnum{5} fois}
%{\blockmoreblocks{Rectangle}
%\blockmove{ajouter \ovalnum{40} à \ovalvariable{x}}
%\blockmove{ajouter \ovalnum{-20} à \ovalvariable{y}}
%}
%\end{scratch}&
%\begin{scratch}
%\initmoreblocks{définir \namemoreblocks{Rectangle}}
%\blockpen{stylo en position d'écriture}
%\blockmove{s’orienter à \ovalnum{90} degrés}
%\blockrepeat{répéter \ovalnum{2} fois}
%{\blockmove{avancer de \ovalnum{40}}
%\blockmove{tourner \turnright{} de \ovalnum{90} degrés}
%\blockmove{avancer de \ovalnum{20}}
%\blockmove{tourner \turnright{} de \ovalnum{90} degrés}}
%\blockpen{relever le stylo}
%\end{scratch}\\
%\textbf{Script principal} &\textbf{Bloc \og rectangle\fg}\\ \hline
%\end{tabularx}
%\end{center}
%
%On rappelle que l'instruction \og s'orienter à 90 \fg{} consiste à s'orienter horizontalement vers la droite. 
%
%\medskip
%
%\textbf{Dans cet exercice, aucune justification n'est demandée}
%
%\medskip

\begin{enumerate}
\item %Quelles sont les coordonnées du point de départ du tracé ?

Le point de départ a pour coordonnées (0~;~0).
\item %Combien de rectangles sont dessinés par le script principal ?
5 rectangles sont dessinés.
\item %Dessiner à main levée la figure obtenue avec le script principal.
On obtient un rectangle le longueur 40 et de largeur 20.
\item 
	\begin{enumerate}
		\item %Sans modifier le script principal, on a obtenu la figure ci-dessous composée de rectangles de longueur $40$ pixels et de largeur $20$ pixels. Proposer une modification du bloc \og rectangle\fg permettant d'obtenir cette figure.

%\begin{center}
%\psset{unit=1cm}
%\begin{pspicture}(4.5,2.8)
%\multirput(0,2)(0.8,-0.4){5}{\psframe(0.4,0.8)}
%\end{pspicture}
%\end{center}
Il suffit d'échanger le 40 et le 20 de \og avancer\fg{} dans le bloc \og Rectangle \fg{}.
		\item %Où peut-on alors ajouter l'instruction \begin{scratch}\blockmove{ajouter \ovalnum{1} à la taille du stylo}\end{scratch} dans le script principal pour obtenir la figure ci-dessous ?
		
%\begin{center}
%\psset{unit=1cm}
%\begin{pspicture}(4.5,2.8)
%\rput(0,2){\psframe[linewidth=1pt](0.4,0.8)}
%\rput(0.8,1.6){\psframe[linewidth=1.5pt](0.4,0.8)}
%\rput(1.6,1.2){\psframe[linewidth=2pt](0.4,0.8)}
%\rput(2.4,0.8){\psframe[linewidth=2.5pt](0.4,0.8)}
%\rput(3.2,0.4){\psframe[linewidth=3pt](0.4,0.8)}
%\end{pspicture}
%\end{center}
Il faut ajouter cette instruction à la fin du \og répéter 5 fois \fg.
	\end{enumerate}
\end{enumerate}