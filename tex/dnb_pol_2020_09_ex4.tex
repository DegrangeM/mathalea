Jean possède 365 albums de bandes dessinées.Afin de trier les albums de sa collection, il les range par série et classe les séries en trois catégories: franco-belges, comics et mangas comme ci-dessous.

\begin{center}
\begin{tabularx}{\linewidth}{|*{3}{X|}}\hline
Séries franco-belges&Séries de comics&Séries de mangas\\ \hline
23 albums \og Astérix \fg&35 albums \og Batman \fg&85 albums \og One-Pièce \fg\\
22 albums \og Tintin\fg&90 albums \og Spider-Man \fg&65 albums \og Naruto \fg\\
45 albums \og Lucky-Luke \fg&&\\ \hline
\end{tabularx}
\end{center}

\medskip
 
Il choisit au hasard un album parmi tous ceux de sa collection.
\medskip

\begin{enumerate}
\item 
	\begin{enumerate}
		\item Quelle est la probabilité que l'album choisi soit un album \og Lucky-Luke\fg ?
		\item Quelle est la probabilité que l'album choisi soit un comics ?
		\item Quelle est la probabilité que l'album choisi ne soit pas un manga ?
	\end{enumerate}
\item Tous les albums de chaque série sont numérotés dans l'ordre de sortie en librairie et chacune des séries est complète du numéro 1 au dernier numéro.
	\begin{enumerate}
		\item Quelle est la probabilité que l'album choisi porte le numéro 1 ?
		\item Quelle est la probabilité que l'album choisi porte le numéro 40 ?
	\end{enumerate}
\end{enumerate}