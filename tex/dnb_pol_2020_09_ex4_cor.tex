%Jean possède 365 albums de bandes dessinées.Afin de trier les albums de sa collection, il les range par série et classe les séries en trois catégories: franco-belges, comics et mangas comme ci-dessous.
%
%\begin{center}
%\begin{tabularx}{\linewidth}{|*{3}{X|}}\hline
%Séries franco-belges&Séries de comics&Séries de mangas\\ \hline
%23 albums \og Astérix \fg&35 albums \og Batman \fg&85 albums \og One-Pièce \fg\\
%22 albums \og Tintin\fg&90 albums \og Spider-Man \fg&65 albums \og Naruto \fg\\
%45 albums \og Lucky-Luke \fg&&\\ \hline
%\end{tabularx}
%\end{center}
%
%\medskip
 
Il choisit au hasard un album parmi tous ceux de sa collection.
\medskip

\begin{enumerate}
\item 
	\begin{enumerate}
		\item %Quelle est la probabilité que l'album choisi soit un album \og Lucky-Luke\fg ?
Il y a 45 albums \og Lucky-Luke\fg{} sur 365 albums en tout ; la probabilité est donc égale à $\dfrac{45}{365} = \dfrac{5 \times 9}{5 \times 73} = \dfrac{9}{73}$.
		\item %Quelle est la probabilité que l'album choisi soit un comics ?
Il y a $35 + 90 = 125$ albums comics sur 365 albums en tout ; la probabilité est donc égale à $\dfrac{125}{365} = \dfrac{5 \times 25}{5 \times 73} = \dfrac{25}{73}$.
		\item %Quelle est la probabilité que l'album choisi ne soit pas un manga ?
Il y a $85 + 65 = 150$ mangas sur 365 albums en tout ; la probabilité de choisir un manga est donc égale à $\dfrac{150}{365} = \dfrac{5 \times 30}{5 \times 73} = \dfrac{30}{73}$.

Donc la probabilité de ne pas choisir un manga est : $1 - \dfrac{30}{73} = \dfrac{43}{73}$.
	\end{enumerate}
\item %Tous les albums de chaque série sont numérotés dans l'ordre de sortie en librairie et chacune des séries est complète du numéro 1 au dernier numéro.
	\begin{enumerate}
		\item %Quelle est la probabilité que l'album choisi porte le numéro 1 ?
Il y a donc 7 albums numérotés 1. La probabilité de choisir un album numéroté 1 est donc $\dfrac{7}{365}$.
		\item %Quelle est la probabilité que l'album choisi porte le numéro 40 ?
IL y a 4 albums numérotés 40, donc la probabilité de choisir un album numéroté 40 est donc $\dfrac{4}{365}$.
	\end{enumerate}
\end{enumerate}
